\documentclass{article}
\usepackage[ngerman]{babel}
\usepackage{amsmath}  % Optional, für mathematische Formeln
\usepackage[table,xcdraw]{xcolor}
\usepackage{graphicx}


\title{Ein Beispiel Dokument} % Title
\author{Fiete Scheel} % Author
\date{\today} % Date


\begin{document}
\maketitle

\tableofcontents

\section{Zeichenformate}
Ich bin ein einfacher Textabschnitt der die Section beschreibt.

\textbf{Ich bin ein fetter Text} und \textit{ich bin ein kursiver Text}.

\section{Mathematische Formeln}
So sehen mathematische Formeln aus.

Eine Formel: \(a^2 + b^2 = c^2\)

Eine Gleichung die total kompliziert istm, wird dann so verwendet.
\[
  E = mc^2
\]

\section{Listen}

\subsection{Ungeordnete Liste}
\begin{itemize}
  \item Erster Punkt
  \item Zweiter Punkt
  \item Dritter Punkt
\end{itemize}

\subsection{Geordnete Liste}
\begin{enumerate}
  \item Erster Punkt
  \item Zweiter Punkt
  \item Dritter Punkt
\end{enumerate}

\section{Tabellen}
\subsection{Einfache Tabelle}
\begin{tabular}{|c|c|}
  \hline
  A & B \\
  \hline
  1 & 2 \\
\hline
\end{tabular}

\subsection{Tabelle mit Headern}
\begin{tabular}{|l|c|c|}
  \hline
  \rowcolor{gray!20} & \textbf{Backend mit Api} & \textbf{Frontend} \\
  \cellcolor{gray!20} \textbf{Basis} & \cellcolor{blue!20} Typescript oder Go & \cellcolor{orange!20}HTML und Javascript \\
  \cellcolor{gray!20} \textbf{Framework} & \cellcolor{blue!20} Nextjs & \cellcolor{orange!20} React \\
  \hline
\end{tabular}

\section{Bilder und Figures}
Das ist ein Beispiel wie ein Bild mit Abstand nach oben und einer Caption eingebunden wird.

\vspace{1cm}
\begin{figure}[h]
  \centering
  \includegraphics[width=0.5\textwidth]{bild.jpg}
  \caption{Ein Bild welches Daten visualisiert und von Ai erstellt wurde}
\end{figure}

\end{document}
