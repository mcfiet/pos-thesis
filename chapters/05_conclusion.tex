\section{Conclusion}

This research provides evidence that confidential computing shows promising potential for production database deployments in multi-cloud environments, albeit with important caveats. Our findings indicate acceptable performance overhead for most workloads, enhanced security guarantees, and reasonable integration with existing cloud-native technologies.
The implementation of our zero-trust architecture—encompassing hardware attestation, encrypted storage, secure networking, and confidential processing—demonstrates the technical feasibility of comprehensive data protection strategies in cloud environments. However, the single-provider limitation and observed performance variability indicate that broader validation across multiple cloud platforms would be necessary before recommending widespread adoption.
Organizations considering confidential computing should carefully evaluate their specific performance requirements and conduct provider-specific testing. As confidential computing technologies continue to evolve, we expect performance characteristics to improve, though mainstream enterprise adoption will depend on addressing current limitations and expanding multi-cloud validation.

\section{Criticism and improvements}

A major criticism of this study is that the underlying infrastructure is based exclusively on virtual machines. Although confidential VMs offer a good opportunity to process sensitive data in the cloud, it should be noted that these are virtualized systems. In contrast to bare-metal servers, where the underlying hardware is used directly, virtual machines share the physical infrastructure with other instances. As a result, performance impacts due to virtualization or resource allocation by the cloud provider cannot be completely ruled out. A direct comparison with bare-metal environments would have been more informative here in order to assess the actual overhead and performance of confidential computing more realistically.

Another aspect is the restriction to a single cloud provider, namely Google Cloud. This means that no generalizable statements can be made about the implementation and performance of confidential VMs across different providers. It would have been desirable to realize a multi-cloud setup in order to evaluate differences between the implementations of different providers. However, this was not possible within the scope of this work, as only Google Cloud provided access to confidential VMs as part of a free test phase. In addition, the different technical implementation of confidential computing at the individual cloud providers makes direct comparability considerably more difficult.
