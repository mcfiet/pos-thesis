In order to protect data holistically, it is considered in three states, each of which requires different security measures:

\textbf{Data at rest} refers to data that is stored on storage media such as hard disks or SSDs, e.g. in databases or file systems. This data can be secured by encryption at storage level (e.g. LUKS, KMS-integrated volumes).

\textbf{Data in transit} includes data that is transferred via networks, for example between microservices, between cloud providers or when accessed by users. Protocols such as TLS or encrypted network topologies (e.g. VPNs, WireGuard) are used here.

\textbf{Data in use} refers to data that is currently being processed in a process or stored in memory. This is traditionally the least protected state, as data must be decrypted in order to be used. This is where Confidential Computing comes in by protecting processing within Trusted Execution Environments (TEEs).

\section{Kuberetes basics}

Kubernetes is an open source platform for orchestrating containerized applications. It enables the automated deployment, scaling and management of applications across clusters. Central concepts are:

\begin{itemize}
\item \textbf{Pods}: The smallest deployable unit, consisting of one or more containers.
\item \textbf{Nodes}: Physical or virtual machines on which pods run.
\item \textbf{Deployments and StatefulStats}: Mechanisms for managing the lifecycle of applications.
\item \textbf{Services}: Abstractions for load balancing and communication between components.
\item \textbf{ConfigMaps and Secrets}: Mechanisms for managing configuration data and sensitive information.
\item \textbf{Persistent Volumes (PV) and Storage Classes}: Management of external storage solutions for stateful applications such as databases.
\item \textbf{CNI (Container Network Interface)}: Plugin-based architecture for configuring network interfaces in Linux containers, enabling pod-to-pod communication across the cluster.
\item \textbf{CSI (Container Storage Interface)}: Standardized interface for storage vendors to develop plugins that work across different container orchestration systems, enabling dynamic provisioning and management of storage volumes.
\end{itemize}

For this project, Kubernetes is used as a platform-independent framework to build a geo-distributed, multi-cloud database infrastructure that integrates aspects such as network encryption, storage security and CPU isolation.

\subsection{Introduction to Confidential Computing}

Confidential Computing describes the approach of protecting sensitive data during processing (data-in-use) through specialized hardware functions that create Trusted Execution Environments (TEEs) at the processor level. The primary technologies include Intel TDX and SGX, AMD SEV variants, and ARM TrustZone \parencite{confidential_computing_consortium_2024}.

This project specifically utilizes AMD SEV-SNP (Secure Encrypted Virtualization - Secure Nested Paging) technology. AMD SEV operates through a dedicated secure processor that manages memory encryption independently of the main CPU cores \parencite{amd_sev_2024, kaplan_amd_encryption_2016}. Each virtual machine receives a unique encryption key, ensuring that memory pages are encrypted with VM-specific keys managed entirely within the secure processor, making them inaccessible to the hypervisor.

Major cloud providers now offer confidential virtual machines (cVMs) that leverage these hardware technologies to provide TEEs protecting against unauthorized access by the hypervisor, operating system, or cloud administrators. In combination with remote attestation mechanisms, these environments can be cryptographically verified as trustworthy before sensitive data is processed. The attestation process involves hardware measurement of the VM's initial state, generation of signed attestation reports, external verification against known good values, and only then provisioning of secrets and sensitive data to verified VMs.

This capability is particularly important for security requirements in regulated industries such as healthcare, finance, and government, where data protection regulations mandate strong controls over data processing environments \parencite{confidential_computing_consortium_2024}.
