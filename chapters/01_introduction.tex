\section{Motivation for multi-cloud databases and encryption at runtime}
With increasing digitalization and globalization, modern companies are faced with the challenge of processing large volumes of sensitive data in a high-performance and secure manner. The use of multi-cloud strategies has established itself as an effective approach to ensure higher availability, geo-redundant resilience and provider-independent scalability. Distributed provision across multiple cloud providers offers a strategic advantage in terms of redundancy and availability, particularly in the area of business-critical databases.

At the same time, there is an increasing need to reliably protect data not only during storage (data-at-rest) or transmission (data-in-transit), but also during processing (data-in-use). Conventional security mechanisms are often not sufficient here, as they do not offer complete protection against threats at infrastructure level or from privileged attackers within the cloud environment.

Confidential computing addresses this problem by using hardware-supported isolation in so-called confidential virtual machines (cVMs). These make it possible to protect sensitive data and processes even from the cloud provider by running them in a protected area of the CPU. The combination of cloud-native databases, distributed Kubernetes clusters and confidential computing thus opens up new possibilities for the secure and highly available operation of data-intensive applications in a multi-cloud infrastructure.

The aim of this project is to evaluate the feasibility and performance of such an architecture, in particular with regard to network and CPU latencies in a realistic, security-focused multi-cloud scenario.

